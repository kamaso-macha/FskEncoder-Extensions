%\externaldocument{ProjectSetup}

The FskEncoder follows the linear MVC model where only two interaction paths are possible:

\begin{enumerate}

	\item between View and Control and vice versa,
	
	\item between Control and Model and vice versa.
	
\end{enumerate}

\begin{figure}[h!]

	\includegraphics[width=\linewidth, keepaspectratio]{./images/ArchitectureOverview.png}
  \caption{Architecture overview}
  \label{fig:architecture-overview}
	
\end{figure}

\begin{minipage}{0.45\textwidth}

	The right-hand side shows the main window with two regions marked with a red frame.
	\\
	The top-one, with the caption \textit{Region}\, is the extension GUI of the input reader extension while the lower one (with the field \textit{File Name}\,) is from target system extension.
	\\
	
	These GUI parts are specific to each extension and are part of the extension packages.

\end{minipage}
\hfill % pushes the two pages to the margins, leaving a gap
\begin{minipage}{0.45\textwidth}

	% use adjustimage to get alignment option, 
	% \includegraphic would push the image out of the top of the page
	
	\adjustimage{width=\textwidth, valign=t}{./images/Mainwindow.png}
	\captionof{figure}{Main widow}
	
	% note that \textwidth will be the width of the minipage
\end{minipage}

\vspace{3 ex}

Figure \ref{fig:architecture-overview}\, \textit{\nameref{fig:architecture-overview}}\, depicts the global structure of the application together with the extensions:

\begin{itemize}

	\item \textbf{View} This is the main window with all GUI elements exept that of the GUI extensions.

	\item \textbf{Control} This is the control for the main window dealing with every thing exept the extension GUIs.

	\item \textbf{Workflow engine} Encapsulates the \textit{business logic}\, of the FskEncoder application and the extensions.

	\item \textbf{Model} The data model of the application. It holds some run-time data and the paroerties of the two configuration files \textit{FskEncoder.properties}\, and \textit{Plugin.properties}\,.

	\item \textbf{Soundplayer} Plays the sound samples on the selected output device with the desired volume.

	\item \textbf{InputReader Extension view} This is the GUI part of the selected input reader extension and \must provide all informations needed to deal with the input format.
	\\
	The example in the figure above shows the GUI of the Ihx8InputReaderExtension which can handle more than one memory section in a single file.

	\item \textbf{InputReader Extension control} Extension specific control which must handle all controls of the extension gui.

	\item \textbf{TargetSystem Extension view} This is the GUI part of the selected target system extension and \must provide all informations needed to deal with the specific protocol format.

	\item \textbf{TargetSystem Extension control} Extension specific control which must handle all controls of the extension gui.

	\item \textbf{TargetSystem Extension protocol} The protocol used to encode the binary data and to convert them into sound samples.

\end{itemize}

Each input reader and target system extension must be fully decoupled from the FskEncoder source (except the interface / base classes needed) and \shall only interact with the application via the provided interfaces.
\\

A deep dive into how to build those extensions is discussed in the section \ref{extension-instantiation}  \textit{\nameref{extension-instantiation}}\, on page \pageref{extension-instantiation}.
\\

By rule, the application itself and each extension lives in it's own .jar file which must contain all the necessary classes but no 3pty libraries (please refer to section \ref{Project-setup}   \textit{\nameref{Project-setup}}\, on page \pageref{Project-setup} for library management).

%===============================================================================

\subsection{Project structure}

This section gives an explanation of the different packages and their purposes. The classes are only discussed if there is a real need for the understanding of building extensions. Most of this information can be found in the JavaDoc of the project.
\\

For each package is specified if its classes and interfaces should not, can or must be used in extensions.
\\

\subsection{Application part}

The following source folder / packages and their contents currently exist:

\begin{samepage}
	\begin{addmargin}[1cm]{0cm}
		\begin{verbatim}

    .
    +---src (folder)
         +---application
         +---control
         |   +---gui
         |   +---validator
         +---extension
         |   +---control
         |   +---encoder
         |   +---factory
         |   +---model
         |   +---protocol
         |   +---sond
         |   +---source
         |   +---target
         |   +---view
         +---model
         +---protocol
         +---sound
         +---view
             +---gui
						
		\end{verbatim}
	\end{addmargin}
\end{samepage}


%===============================================================================

\subsection{Packages of the 'src' folder}

%*******************************************************************************

\subsubsection{application}

\private

Contains the application class together with classes to handle cli parameter, some 'hard coded' properties and for program exit.

%*******************************************************************************

\subsubsection{control}

\private

Contains the controller logic for work-flow and background task execution.


%*******************************************************************************

\subsubsection{control.gui}

\private

Held all controllers for the basic GUI panels.

%*******************************************************************************

\subsubsection{control.validator}

\private

Contains various validators for the GUI elements.

%*******************************************************************************

\subsubsection{extension}

\public

This package provides the logic needed for extending the application. A deeper discussion can be found in section \ref{api}  \nameref{api} \, on page \pageref{api}.
\\
The content of this package is packed in a seperate file \textit{FskEncoderExtension.jar} \, during build process and \must be used as library in the extension part.


%*******************************************************************************

\subsubsection{extension.control}

\public

Contains classes and interfaces needed for extension controllers.


%*******************************************************************************

\subsubsection{extension.encoder}

\public

This is more a library of basic functionallity and contains all (currently) necessary classes for sound sample generation.
\\

Protocol specific encoders \must be placed in the package of the target system extension protocol. 
\\

Please refer to the \textit{Mpf1BitEncoder}\, class in the \textit{Microprecessor 1}\, extension to learn how to use the basic methods to build a complex protocol.


%*******************************************************************************

\subsubsection{extension.factory}

\public

This package is the place for the base classes and interfaces needed to build a extension factory.


%*******************************************************************************

\subsubsection{extension.model}

\public

This package contains all classes and interfaces of the data model needed to implement a input reader extension.


%*******************************************************************************

\subsubsection{extension.protocol}

\public

This package contains all classes and interfaces needed to implement a target system extension protocol.


%*******************************************************************************

\subsubsection{extension.sound}

\partially

This package contains the definition of the audio format of the FskEncoder soundplayer and is required in an extension protocol.


%*******************************************************************************

\subsubsection{extension.source}

\public

This package contains all classes and interfaces needed to implement a input reader extension.


%*******************************************************************************

\subsubsection{extension.target}

\public

This package is currently empty.


%*******************************************************************************

\subsubsection{extension.view.gui}

\public

Holds all base classes for extension guis.


%*******************************************************************************

\subsubsection{model}

\private

This package contains all classes of the internal data model.


%*******************************************************************************

\subsubsection{protocol}

\public

This package is currently empty.


%*******************************************************************************

\subsubsection{sound}

\private

All stuff around the sound player.


%*******************************************************************************

\subsubsection{view}

Currently empty.


%*******************************************************************************

\subsubsection{view.gui}

\private

Contains all classes needed for the main window and the basic GUI panels.


%===============================================================================

\subsection{Extensions part}

In the extension part are all packages located, which are related to extensions. They \must be strictly kept apart of the application part!

The following source folder / packages and their contents currently exist:

\begin{samepage}
	\begin{addmargin}[1cm]{0cm}
		\begin{verbatim}

    .
    +---src (folder)
        +---source
        |   +---bin
        |   +---ihx
        |       +---x8
        +---target
            +---microprofessor1
            +---z80trainer
						
		\end{verbatim}
	\end{addmargin}
\end{samepage}


%===============================================================================

\subsubsection{source.bin}

\private

Contains all sources for the binary input file reader.


%*******************************************************************************

\subsubsection{source.ihx}

\partially

Contains all common used sources for the Intel-Hex input file reader.


%*******************************************************************************

\subsubsection{source.x8}

\private

Contains the 8-bit specific implementation of the Ihx-reader extension.


%*******************************************************************************

\subsubsection{target.microprofessor1}

\private

Contains the classes of the Microprofessor 1 extension.


%*******************************************************************************

\subsubsection{target.z80trainer}

\private

Contains the classes of the Z80 trainer extension.



