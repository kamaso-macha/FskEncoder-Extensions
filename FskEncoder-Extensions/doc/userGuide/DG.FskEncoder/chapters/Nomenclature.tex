\subsection{Keywords}

The key words \footnote{Taken from IETF RFC 2119, Key words for use in RFCs to Indicate Requirement Levels}
"MUST", "MUST NOT", "REQUIRED", "SHALL", "SHALL NOT", "SHOULD", "SHOULD NOT", "RECOMMENDED", "MAY", and
"OPTIONAL" in this document are to be interpreted as:

\begin{enumerate}

	\item MUST This word, or the terms "REQUIRED" or "SHALL", mean that the
definition is an absolute requirement of the specification.

	\item  MUST NOT This phrase, or the phrase "SHALL NOT", mean that the
definition is an absolute prohibition of the specification.

	\item  SHOULD This word, or the adjective "RECOMMENDED", mean that there
may exist valid reasons in particular circumstances to ignore a
particular item, but the full implications must be understood and
carefully weighed before choosing a different course.

	\item  SHOULD NOT This phrase, or the phrase "NOT RECOMMENDED" mean that
there may exist valid reasons in particular circumstances when the
particular behavior is acceptable or even useful, but the full
implications should be understood and the case carefully weighed
before implementing any behavior described with this label.

	\item MAY This word, or the adjective "OPTIONAL", mean that an item is
truly optional. One vendor may choose to include the item because a
particular marketplace requires it or because the vendor feels that
it enhances the product while another vendor may omit the same item.
An implementation which does not include a particular option MUST be
prepared to inter-operate with another implementation which does
include the option, though perhaps with reduced functionality. In the
same vein an implementation which does include a particular option
MUST be prepared to inter-operate with another implementation which
does not include the option (except, of course, for the feature the
option provides.)

\end{enumerate}


%===============================================================================

\subsection{UML grapics}

This document contains some UML class diagrams. In this diagrams are two styles used to draw the classes: 

\begin{itemize}

	\item	First the standard style, white background with attributes and methods defined for all classes / interfaces in the package of interest and
	
	\item second the greyed style, grey background and dashed lines, without definition of attributes and methods for all collabrating classes / interfaces of different packages.
	
\end{itemize}

The greyed classes / interfaces are included for completeness and to lay out the dependencies between packages.
\\

Arrows in the relationship of classes and interfaces and their meanign :
\\

\begin{addmargin}[1cm]{0cm}
	\begin{tikzpicture}[yscale=-1, y=3ex]

		\draw [-{Latex}]							(0, 1) to ++(1, 0) node [right] {\quad Association};
		\draw [-{Diamond[open]}]			(0, 2) to ++(1, 0) node [right] {\quad Aggregation};
		\draw [-{Diamond}]						(0, 3) to ++(1, 0) node [right] {\quad Composition};
		\draw [-{Latex[open]}]				(0, 4) to ++(1, 0) node [right] {\quad Generalization};
		\draw [-{Latex[open]},dashed]	(0, 5) to ++(1, 0) node [right] {\quad Realization};
		\draw [-{Stealth},dashed]			(0, 6) to ++(1, 0) node [right] {\quad Dependency / Usage};
		
	\end{tikzpicture}
\end{addmargin}

\vspace{2ex}

If two classes A and B have a relationship, than

\begin{description}[style=multiline,leftmargin=8em]
	
	\item[association] between A and B means that A hold a reference to B so that A can communicate with B,
	
	\item[aggregation] means, that A holds one or more refeerences to B,
	
	\item[composition] means, that B is part of A in that way, that B can't exist without A. 
	
	\item[generalization] is, when B extends A
	
	\item[realization] is, when A implements the interface B
	
	\item[dependency] means, that A uses B but without keeping a reference to B.

\end{description}


